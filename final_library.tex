\documentclass{article}
\usepackage{amsmath,amssymb}
\usepackage[margin=3cm]{geometry}
\usepackage{graphicx}

\newcounter{zone}
\setcounter{zone}{0}
\newcommand{\zone}{\clearpage\refstepcounter{zone}\section*{Zone \arabic{zone}}}
\newcounter{question}
\setcounter{question}{0}
\newcounter{variant}
\newcounter{questionpoints}
\newcommand{\question}[1]{\newpage \refstepcounter{question} \setcounter{variant}{0} \setcounter{questionpoints}{#1}}
\newcommand{\variant}{\vspace{4em}\refstepcounter{variant}\noindent \arabic{question}/\arabic{variant}. (\arabic{questionpoints} point\ifnum \thequestionpoints > 1 s\fi) }
\newenvironment{answers}{\begin{enumerate}}{\end{enumerate}}
\newcommand{\answer}{\item }
\newcommand{\correctanswer}{\item $\bigstar$ }
\renewcommand{\theenumi}{\Alph{enumi}}

\begin{document}

\begin{center}
\textbf{\Large Final Exam}
\end{center}

\bigskip
\noindent
\begin{itemize}
\item There are {\bf 3} problems worth points as shown in each question.
\item You must not communicate with other students during this test.
\item No books, notes, \textbf{calculators}, or electronic devices allowed.
\item This is a 20 minute exam.
\item Do not turn this page until instructed to.
\item There are several different versions of this exam.
\end{itemize}

\bigskip\bigskip
\noindent
\textbf{\Large 1. Fill in your information:}

\bigskip
{\Large\bf
\begin{tabular}{ll}
Full Name: & \underbar{\hskip 8cm} \\[0.5em]
UIN (Student Number): & \underbar{\hskip 8cm} \\[0.5em]
NetID: & \underbar{\hskip 8cm}
\end{tabular}
}

\bigskip
\noindent
\textbf{\Large 3. Fill in the following answers on the Scantron form:}

%%%%%%%%%%%%%%%%%%%%%%%%%%%%%%%%%%%%%%%%%%%%%%%%%%%%%%%%%%%%%%%%%%%%%%
%%%%%%%%%%%%%%%%%%%%%%%%%%%%%%%%%%%%%%%%%%%%%%%%%%%%%%%%%%%%%%%%%%%%%%
\zone

%%%%%%%%%%%%%%%%%%%%%%%%%%%%%%%%%%%%%%%%%%%%%%%%%%%%%%%%%%%%%%%%%%%%%%
\question{1}

\variant
Evaluate the series $\displaystyle \sum_{n=0}^\infty
\frac{3^{n+1}}{4^n}.$
\begin{answers}
\answer $1/4$
\correctanswer $12$
\answer $3/4$
\answer $4$
\answer $36$
\end{answers}

\variant
Evaluate the series $\displaystyle \sum_{n=0}^\infty
\frac{3^{n+1}}{5^n}.$
\begin{answers}
\answer $2/5$
\answer $5$
\answer $1/5$
\correctanswer $15/2$
\answer $10$
\end{answers}

\variant
Evaluate the series $\displaystyle \sum_{n=0}^\infty
\frac{2^{n+1}}{3^n}.$
\begin{answers}
\answer $3$
\answer $1/3$
\answer $2/3$
\correctanswer $6$
\answer $12$
\end{answers}

%%%%%%%%%%%%%%%%%%%%%%%%%%%%%%%%%%%%%%%%%%%%%%%%%%%%%%%%%%%%%%%%%%%%%%
\question{1}

\variant
Find the average value of the function $f(x) = \displaystyle x\,
\sqrt{2x^2+1}$ on the interval $[0,2]$.
\begin{answers}
\correctanswer $\frac{13}{6}$
\answer $\frac{\sqrt2}{3}$
\answer $\frac{\sqrt2}{6}$
\answer $\frac{13}{3}$
\answer $\frac{26}{3}$
\end{answers}

\variant
Find the average value of the function $f(x) = \displaystyle x\,
\sqrt{5x^2-4}$ on the interval $[1,2]$.
\begin{answers}
\correctanswer $\frac{21}{10}$
\answer $\frac{21}{5}$
\answer $\frac{2\sqrt2}{15}$
\answer $\frac{\sqrt2}{15}$
\answer $21$
\end{answers}

\variant
Find the average value of the function $f(x) = \displaystyle
\sqrt{8x+1}$ on the interval $[1,3]$.
\begin{answers}
\correctanswer $\frac{49}{12}$
\answer $\frac{3\sqrt3-1}{24}$
\answer $\frac{3\sqrt3-1}{12}$
\answer $\frac{49}{6}$
\answer $\frac{98}{3}$
\end{answers}

%%%%%%%%%%%%%%%%%%%%%%%%%%%%%%%%%%%%%%%%%%%%%%%%%%%%%%%%%%%%%%%%%%%%%%
\question{1}

\variant
Evaluate $\displaystyle\int_{\pi}^{2 \pi} x^2\cos x \, dx$.
\begin{answers}
\answer $-6\pi$
\answer $-2\pi$
\answer $0$
\answer $2\pi$
\correctanswer $6\pi$
\end{answers}

\variant
Evaluate $\displaystyle\int_{\pi/2}^{3\pi/2} x^2\sin x \, dx$.
\begin{answers}
\correctanswer $-4\pi$
\answer $-8\pi$
\answer $8\pi$
\answer $0$
\answer $4\pi$
\end{answers}

\end{document} 